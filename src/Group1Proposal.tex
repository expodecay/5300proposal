%! Author = Rick Ramirez/Ruben Bramasco/Vikram Sai Kishan Sriram
%! Date = 9/5/2021

% Preamble
\documentclass[11pt]{article}
\title{Computer Science 5300
\partition\
    Advanced Algorithm Design and Analysis\\
    \vspace{1ex}
    Group Proposal}
\author{\small }
\date{Fall 2021}
\linespread{1.3}
\newcommand{\partition}{\rule{\linewidth}{0.8pt}}

% Packages
\usepackage{amsmath}
% my own titles
\makeatletter
\renewcommand{\maketitle}{
    \begin{center}
        \@date \hfill  \@author\\
        {\Large \textsc{\@title}}
        \partition\\
    \end{center}
}


% Document
\begin{document}
	\vspace*{-3cm}
	{\let\newpage\relax\maketitle}
	    
    \section{\normalsize What style your use, IEEE or MLA?}
        IEEE
    \section{\normalsize Title of the topic}
        Branch and Bound
    \section{\normalsize Group number and full names of all members.}
        Group \#1: Rick Ramirez, Ruben Bramasco, Vikram Sai Kishan Sriram
    \section{\normalsize A brief description of the topic}
    	The branch and bound framework has been found to be a useful method of solving combinatorial optimization problems by means of a state space search and has become a commonly used tool for solving NP-hard optimization problems in general. The technique involves forming a set of candidate solutions in the form of a rooted tree. Each branch is explored and checked against an estimated bound on the optimal solution. If the branch does not yield a better solution, it is discarded. 
    	
    	This design paradigm has applications in many areas of physics and economics, and is actively being improved upon. We examine new classical applications of the branch and bound method involving evacuation techniques and calculating the maximum edge-weighted clique problem, as well as explore a quantum adaptation of the technique along with it's application to the ising model.
    \section{\normalsize Live or pre-recorded? For your team presentation.}
        Pre-recorded
    \section{\normalsize At least five main references to recent books or technical articles you plan to use}

    \nocite{montanaro2020quantum, baccari2020verifying, san2019new, goerigk2013branch, cormen2009introduction}
    \bibliography{main}
    \bibliographystyle{plain}
\end{document}
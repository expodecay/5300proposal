%! Author = rickr
%! Date = 11/17/2021

\subsection{Pruning Rules}
	While the use of a heuristic allows us to approximate the utility of a state without performing a complete search, pruning rules allow us to neglect portions of the tree without affecting the final choice. 
	Many of the pruning techniques are developed in the context of Artificial Intelligence, however the basic function holds across all applications involving combinatorial optimization. 
	\subsubsection{Lower-Bound}
		Producing a lower bound on the objective function at each node is the most common way to prune a search space \cite{morrison2016branch}. 
		The lower bound then used to prune sub-problems who's lower bound is no better than the incumbent solution. 
		The value of the lower bound is calculated by relaxing aspects of the problem, and as such, some lower bound calculations may be easier to determine than others. 
		The general technique is to attempt pruning with a lower bound that is easy to calculate before moving on to more complex (but possibly tighter) lower bounds. 
	\subsubsection{Alpha-Beta Pruning}
		Alpha-beta pruning is often used in areas of artificial intelligence that employ a MinMax decision rule for minimizing the possible loss of a worst case scenario. 
		The idea is to maximize the minimum gain, however, issues arise when the number of game states to examine is exponential in the depth of the tree. 
		Alpha-beta pruning, while not eliminating the exponent, effectively cuts it in half by assigning a numerical score to each terminal node \cite{russell2010artificial}. 
		Each terminal node represents the outcome of the player with the next move.
		The technique is referred to as an adversarial search algorithm because the game tree can represent many two-player zero-sum games. 
		Two values $\alpha$ and $\beta$ represent the minimum score that the maximizing player is assured of, and maximum score that the minimizing player is assured of. 
		When the value of a terminal node results in $\beta$ being less than $\alpha$, the maximizing player may ignore descendants of the node.  
	\subsubsection{Decision Tree Pruning}
		For some problems, it is found that decision-tree learning algorithms can generate large trees with no inherent pattern. This can happen when too many irrelevant features are fed into the training model. 
		For example, when measuring the outcome of a tossed coin, the color, time of day, the exact person doing the flipping, have no effect on the outcome. 
		However, if the model is provided these features as input, then the training could be subject to over-fitting. 
		In general, over-fitting is more likely to occur when the hypothesis space and number of input attributes grows \cite{russell2010artificial}. 
		To combat this, we use decision-tree pruning. 
		The general idea is that we start with a full tree and look at a test node whose descendants are only leaf nodes. 
		If testing detects only noise in the data, then we eliminate the node and replace it with the leaf node. 
	
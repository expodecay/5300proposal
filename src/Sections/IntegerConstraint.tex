%! Author = rickr
%! Date = 11/17/2021
\section{Integer Constraint Problems}
    One of the many types of problems that can be solved by 
    branch-and-bound are Integer Constraint Problems. 
    Integer Constraint Problems at their core are combinatorial 
    optimization problems, thus can be solved by branch-and-bound.
    For this example we will use a system of linear equations problem,
    apply an integer constraint, and show how it can be solved using 
    branch-and-bound. In the example below we focus on the case of 
    two equations, but this can be expanded and solved by branch-and-bound.
    \\ 
    \noindent
	\textbf{\underline{System of Linear Equations Problem}:}\\
	\underline{Instance}: \\ $Z = f(x,y)$  \\ $h(x,y) \leq c_1$ \\ $g(x,y) \leq c_2 $ \\
	\underline{Question}: What is the best assignment of $x$ and $y$ such that $Z$ is maximized?
    \\

        \subsubsection{Creating an Upper-Bound}
        First thing we must notice about Integer Constraint problems is that
        their solution space is smaller than their continuous counterparts and 
        is in fact a subset.
        We can exploit this and create an upper-bound. To create the upper-bound
        we will relax the integer constraint and solve the relaxed problem.
        Solving the system of equations we get some optimal values: $x^{'}$, $y^{'}$, 
        and $Z^{'}$. We notice that the integer constraint solution
        space is a subset of the relaxed solution space, therefore the largest
        possible value of $Z$ when applying the integer constraint is 
        $ Z = \lfloor Z^{'} \rfloor$. This will now be our upper-bound ($UP$), which can be used
        to find the optimal solution.

        \subsubsection{Combinations}
        Now we need to turn our problem into a combinatorial optimization problem.
        To do this, we need to notice that, graphically, the solution in the continuous space is at 
        the intersection of the two lines.\\
        
        \begin{figure}[H]
            \begin{center}
                \begin{tikzpicture}
                    \begin{axis}[xlabel=$x$,ylabel=$y$,
                    xmin=0,xmax=7,ymin=0,ymax=7, axis lines=center, axis equal]
                    
                        \addplot[domain=-10:10, color=black,]{(45/6)-(5/6)*x};
                        
                        \addplot[domain=-10:10, color=black,]{(42/7) - (4/7)*x};
                    
                    \end{axis}
                \end{tikzpicture}
            \end{center}
            \caption{An example of a system of linear equations.}
        \end{figure}
        
        Therefore, when applying the integer constraint our
        solution will be near the intersection. To generate all the solution combinations we will 
        use two operations, the floor ($\lfloor \rfloor$) and ceiling ($\lceil \rceil$) operation. 
        For example, take the solution pair in the continuous space $x^{'}$ and $y^{'}$,
        applying the floor and ceiling to each, we get the candidate integer constraint solutions:  
        $(\lceil x^{'} \rceil, \lceil y^{'} \rceil)$, $(\lfloor x^{'} \rfloor, \lceil y^{'} \rceil)$
        $(\lceil x^{'} \rceil, \lfloor y^{'} \rfloor)$, $(\lfloor x^{'} \rfloor, \lfloor y^{'} \rfloor)$.
        Each floor and ceil operation that we do will be a new constraint or equation on the problem,
        and can be represented in the following form $x \leq c_i$ or $x \geq c_i$. 
        Each time a new branch is created every previous constraint on that path is 
        considered in the calculation of its variables.

        \subsubsection{Algorithm}
        There are many ways to solve this problem using branch-and-bound.
        One way is to make an algorithm that utilizes a system of equations solver in the continuous space.
        Then at each branch apply a new constraint/equation, and have the solver compute the maximum value.

        
%! Author = rickr
%! Date = 11/17/2021

\section{TSP Better Lower Bound}
    There are problems like the traveling salesman problem in this example
    that are easy to model using branch-and-bound but do not see too much improvements.
    The reason BnB does not work well with TSP is because BnB relies on psudeo
    parrallelism, BnB will choose to expand best possible path each time, and 
    with a problem like TSP, the best possible path will almost always be the 
    last expanded path. This makes the algorithm almost useless since you 
    would expand almost every combination of tour, and esentially brute-force.
    To combat this researchers have used heuristics, in hopes of making a better
    choice for a lower bound.

    \subsubsection{Creating a Better Lower Bound}
    Firstly, our lower bound calculation before our heuristic is simply the sum 
    of the tour edges. To improve this we will utilize an encoding scheme known as 
    a graph adjacency matrix, and reduce it. There are two reasons for the reduction, 
    the first is that reduction keeps track of our current state of the tour, and the second
    is that the cost of reduction is an admissiable heuristic in the TSP problem.
    After computing the reduction cost we can now make our lower bound calculation.
    \begin{equation}
        LB = LB_p + w(p, n) + R(M_n)
    \end{equation}
    Where $LB_p$ is the lower-bound of the parent, $w(p, n)$ is the weight of the edge
    between the parent, and $R(M_n)$ is the cost of reduction. 
    
    It is important to realize that branch and bound is not one size fits all algorithm, so
    although there maybe a problem that can be encoded with an adjacency matrix, it does not mean
    there exist an admissiable heuristic for it. 

   
    \begin{vmatrix}
        $\infty$ & 9 & 2 & 8\\
        9 & $\infty$ & 7 & 3 \\
        2 & 7 & $\infty$ & 4 \\
        8 & 3 & 4 & $\infty$
    \end{vmatrix}
        %!c = 2 + 3 + 2 + 3
    \begin{vmatrix}
        $\infty$ & 7 & 0 & 6\\  
        6 & $\infty$ & 4 & 0 \\
        0 & 5 & $\infty$ & 2 \\
        5 & 0 & 1 & $\infty$
    \end{vmatrix}

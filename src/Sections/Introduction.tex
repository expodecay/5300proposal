%! Author = rickr
%! Date = 11/17/2021

\section{Introduction}
	The branch-and-bound design paradigm has proven exceedingly useful in solving combinatorial optimization problems with applications in artificial intelligence, applied mathematics and theoretical computer science. 
	The design can be applied to any problem where the goal is to find a minimum-cost solution in a setting where one has access to a bounding function that returns a lower bound on a given subset of solutions and a branching rule that can be applied to partition a subset if possible solutions cannot be ruled out. 
	The technique has been used in solving a number of NP-hard problems and may also serve as the base of various heuristics.\\
	
	The success of branch-and-bound has also sparked interest in applications involving quantum computation by encoding optimization problems to the ground state of classical spin systems. 
	This has led to the development of quantum branch-and-bound algorithms and have been shown to increase performance significantly. 
	
	
	